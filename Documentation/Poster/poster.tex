\documentclass[final,dvipsnames]{beamer}

% ====================
% Packages
% ====================

\usepackage[T1]{fontenc}
\usepackage{tgbonum}
\usepackage[size=a0,orientation=portrait]{beamerposter}
\usetheme{gemini}
\usecolortheme{mit}
\usepackage{graphicx}
\usepackage{booktabs}
\usepackage{tikz}
\usetikzlibrary{trees, matrix, positioning, patterns, shapes, shadows, shapes.arrows, arrows.meta, shapes.multipart, decorations.pathreplacing, calc, tikzmark, shapes.geometric}
\usepackage{pgfplots}
\pgfplotsset{compat=1.14}
\pgfplotsset{every axis/.append style={very thick}}
\usepackage{pgf-pie}
\usepackage[binary-units]{siunitx}
\usepackage{anyfontsize}
\usepackage{amsmath}
\usepackage{mathdots}
\usepackage{mathtools}
\usepackage{bm}
\usepackage{ulem}

% ====================
% Lengths
% ====================

% If you have N columns, choose \sepwidth and \colwidth such that
% (N+1)*\sepwidth + N*\colwidth = \paperwidth
\newlength{\sepwidth}
\newlength{\colwidth}
\setlength{\sepwidth}{0.025\paperwidth}
\setlength{\colwidth}{0.3\paperwidth}

\newcommand{\separatorcolumn}{\begin{column}{\sepwidth}\end{column}}

% ====================
% Title
% ====================

\title{\sout{Safe}Tix QR-Code Scanner 2.0}

\author{Julian \inst{1} \and Alexander Mattingley-Scott \inst{1}}

\institute[shortinst]{\inst{1} Heidelberg University}

% ====================
% Footer (optional)
% ====================

\footercontent{
	Binary Hacking 2024/25 \hfill
	\href{mailto:your.email@stud.uni-heidelberg.de}{your.email@stud.uni-heidelberg.de}}
% (can be left out to remove footer)

% ====================
% Logo (optional)
% ====================

% use this to include logos on the left and/or right side of the header:
%\logoright{\includegraphics[height=6cm]{}}
%\logoleft{\includegraphics[height=6cm]{}}

% ====================
% Body
% ====================

\begin{document}

\begin{frame}[t, fragile]
\begin{columns}[t]
\separatorcolumn

\begin{column}{\colwidth}

	\begin{block}{Motivation and Challenges}

		\begin{itemize}
			\item Explain your problem here. 
			\item Also motivate why you looked into the specific problem.
		\end{itemize}

	\end{block}


\end{column}

\separatorcolumn

\begin{column}{\colwidth}

	\begin{alertblock}{Methodology and Concept}

		How did you solve the problem.

		\begin{enumerate}
				\item First do this.
				\item Then this.
				\item Then that.
		\end{enumerate}

		This solves the problem nicely! Citation\cite{goli_resprop_2020}

	\end{alertblock}

	\begin{block}{Details about X}

		Give some details about a specific part of your project you think is
		important or interesting.

	\end{block}

	\begin{block}{Details about Y}

		Same for another part.

	\end{block}

\end{column}

\separatorcolumn

\begin{column}{\colwidth}

	\begin{block}{Results}

		Show some of your results. For reversing tasks: show what you found about
		the target. For exploitation tasks: show to what results your topic can be
		used.

	\end{block}

	\begin{exampleblock}{Future Work}
		How would you improve on your work in the future?
		\begin{itemize}
			\item Make everything cleaner.
			\item Make everything faster.
			\item Make everything cheaper.
		\end{itemize}
	\end{exampleblock}

	\begin{block}{References}
		%\nocite{*}
		\footnotesize{\bibliographystyle{ieeetr}\bibliography{poster.bib}}
	\end{block}

\end{column}

\separatorcolumn
\end{columns}
\end{frame}

\end{document}
